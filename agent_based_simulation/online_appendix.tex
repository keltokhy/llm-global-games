\documentclass[10pt]{article}
\usepackage{amssymb,amsmath,amsthm}
\usepackage[margin=1in]{geometry}
\usepackage[hidelinks]{hyperref}
\usepackage{graphicx}
\usepackage{booktabs}
\usepackage{float}

\urlstyle{same}
\interfootnotelinepenalty=10000

\title{Online Appendix:\\LLMs Can Play (Global) Games}
\author{Khaled Eltokhy}
\date{}

\begin{document}
\maketitle

\section{Decomposition: Which Channel Drives the Stability Effect?}

The stability design manipulates three channels simultaneously (direction, clarity, dissent). To determine which drives the effect, I run three single-channel treatments, each activating only one manipulation while holding the other two at baseline.

\begin{figure}[t]
  \centering
  \includegraphics[width=0.9\linewidth]{figures/fig11_decomposition.pdf}
  \caption{Single-channel decomposition of the stability design. Each panel shows the treatment effect $\Delta(\theta)$ for one channel in isolation.}
\end{figure}

The direction channel produces the largest average treatment effect at $+2.4$~pp, with effects concentrated near $\theta^*$. The dissent channel contributes $+1.6$~pp uniformly. The clarity channel produces only $+0.3$~pp with a non-monotone pattern.

\begin{table}[t]
\centering
\caption{Single-channel decomposition of the stability design (primary model: Mistral Small Creative).}
\label{tab:decomposition}
\small
\begin{tabular}{lccc}
\toprule
Channel & Mean & $r$ & $\Delta$ \\
\midrule
Full stability & 0.319 & $-0.626$ & -0.094 \\
Clarity only & 0.126 & $-0.857$ & -0.287 \\
Direction only & 0.148 & $-0.826$ & -0.266 \\
Dissent only & 0.140 & $-0.837$ & -0.274 \\
\midrule
Sum of channels & --- & --- & -0.827 \\
Full design & --- & --- & -0.094 \\
\bottomrule
\end{tabular}
\vspace{0.25em}
\footnotesize\emph{Notes:} Each row is a separate infodesign run for Mistral Small Creative on the same $\theta$ grid as Table~\ref{tab:infodesign_summary}. $\Delta$ reports the mean difference vs.\ the baseline infodesign mean (Table~\ref{tab:infodesign_summary}).
\end{table}


The sum of single-channel effects is $+4.3$~pp, far smaller than the full stability design effect ($+19.5$~pp). This implies strong complementarities: combining ambiguity (clarity), softened direction, and mixed-valence cues shifts behavior much more than the sum of each channel in isolation.

\section{Robustness Details}

\subsection{Agent Count Variation}

I vary the number of agents per period ($n \in \{5, 10, 25, 50, 100\}$) using Mistral Small Creative. The correlation is stable: $r = +0.60$ ($n = 5$), $r = +0.63$ ($n = 10$), $r = +0.67$ ($n = 25$), $r = +0.65$ ($n = 50$), $r = +0.65$ ($n = 100$). The slight increase from $n = 5$ to $n = 25$ likely reflects reduced discretization noise.

\subsection{Network Topology}

I compare the baseline communication network ($k = 4$) with a denser network ($k = 8$). The denser network produces $r = +0.66$ (vs.\ $+0.68$ for $k = 4$), with a slightly lower mean join rate of 0.41 (vs.\ 0.45). Additional contacts do not substantially amplify coordination.

\subsection{Mixed-Model Games}

A five-model mixed-population game produces $r = +0.77$ (pure) and $r = +0.75$ (communication)---if anything, higher than single-model correlations. Equilibrium alignment is not an artifact of model homogeneity.

\subsection{Bandwidth Sensitivity}

\begin{table}[t]
\centering
\caption{Bandwidth robustness: treatment effects $\Delta$ (treatment $-$ baseline) within each bandwidth condition (primary model: Mistral Small Creative). Top row shows baseline join rates for reference.}
\label{tab:bandwidth}
\small
\begin{tabular}{lccc}
\toprule
 & BW=0.05 & BW=0.15 & BW=0.30 \\
\midrule
Baseline (level) & 0.054 & 0.409 & 0.061 \\
\midrule
\multicolumn{4}{l}{\textit{Treatment effect $\Delta$ (treatment $-$ baseline):}} \\
Stability & +0.007 & -0.090 & +0.009 \\
Upper cens. & +0.062 & -0.031 & +0.053 \\
Lower cens. & +0.101 & -0.019 & +0.096 \\
\bottomrule
\end{tabular}
\end{table}


Qualitative treatment effects are robust across bandwidths, though magnitudes vary---especially for the stability design, whose effect peaks at the baseline bandwidth. The baseline bandwidth of 0.15 is approximately optimal for detecting treatment effects on the experimental grid.

\subsection{Cross-Model Replication of Information Design}

\begin{figure}[t]
  \centering
  \includegraphics[width=0.95\linewidth]{figures/fig14_cross_model_infodesign.pdf}
  \caption{Cross-model replication of information design treatments. Each panel shows join fraction vs.\ $\theta$ for one model under baseline, stability, scramble, and flip conditions.}
  \label{fig:oa_crossmodel_infodesign}
\end{figure}

Table~\ref{tab:crossmodel} reports cross-model replication of information design treatments. The flip inversion replicates across all models tested ($r > +0.43$ for all six). The scramble test shows more heterogeneity: Mistral, GPT-OSS, and Qwen3 235B show clean collapse ($r \approx 0$), but Llama 3.3 70B and Ministral 3B retain baseline-level correlations under scramble ($r = -0.81$ and $r = -0.66$), suggesting these models extract signal from features the scramble does not disrupt (e.g., within-country narrative coherence). OLMo shows an attenuated baseline relationship, while Qwen3 30B shows a large reduction in correlation under scramble and a clear flip effect.

\begin{table*}[t]
\centering
\caption{Cross-model replication of key information design conditions. $r$ is the correlation between $\theta$ and join fraction.}
\label{tab:crossmodel}
\small
\begin{tabular}{lcccccc}
\toprule
& \multicolumn{2}{c}{Baseline} & \multicolumn{2}{c}{Scramble} & \multicolumn{2}{c}{Flip} \\
\cmidrule(lr){2-3} \cmidrule(lr){4-5} \cmidrule(lr){6-7}
Model & Mean & $r$ & Mean & $r$ & Mean & $r$ \\
\midrule
Mistral Small Creative & --- & --- & --- & --- & --- & --- \\
GPT-OSS 120B & 0.127 & $-0.801$ & 0.132 & $+0.080$ & 0.677 & $+0.754$ \\
Llama 3.3 70B & 0.107 & $-0.809$ & 0.105 & $-0.810$ & 0.887 & $+0.717$ \\
Ministral 3B & 0.220 & $-0.632$ & 0.118 & $-0.658$ & 0.804 & $+0.847$ \\
Qwen3 30B & 0.247 & $-0.612$ & 0.279 & $-0.119$ & 0.784 & $+0.848$ \\
Qwen3 235B & 0.399 & $-0.878$ & 0.394 & $-0.020$ & 0.430 & $+0.871$ \\
\bottomrule
\end{tabular}
\end{table*}


\subsection{Information Design with Communication}

Under baseline information design with communication, mean join rate is 3.0\% ($-9.3$~pp vs.\ baseline without communication). Communication \textit{reduces} join rates across all information design conditions---superficially counterintuitive, but resolved by noting that additional information from others' messages tends to reveal that the regime is strong (the $\theta$ grid extends upward from $\theta^*$).

\section{Implementation Details}

\subsection{LLM Decoding Parameters}

All LLM calls use temperature $= 0.7$ and \texttt{max\_tokens} $= 512$. Each agent produces a single sample per decision (no majority voting or averaging). The API provider is OpenRouter; all calls use the same endpoint and parameters across models.

\subsection{Prompt Templates}

\paragraph{Decision prompt (pure treatment).}
\begin{quote}
\small
\texttt{You are a citizen deciding whether to JOIN an uprising or STAY home.}

\texttt{You only have your private briefing below. You do not know how many others will JOIN.}

\texttt{Start your first line with exactly JOIN or STAY, then give 1--2 concise sentences of reasoning.}
\end{quote}

\paragraph{Communication prompt.}
\begin{quote}
\small
\texttt{You are a citizen in a country with an authoritarian regime. You have received a private intelligence briefing. You are now in contact with a small number of trusted people. Based on your briefing, write a brief message (1--2 sentences) to share your read of the situation. Be natural---you're talking to people you trust.}
\end{quote}

\paragraph{Surveillance communication prompt.} Identical to the communication prompt, with the addition:
\begin{quote}
\small
\texttt{IMPORTANT: You have reason to believe that your communications are being monitored by regime security services. Messages deemed subversive could have serious consequences for you and your contacts.}
\end{quote}

\subsection{Randomization}

Each country--period draws $\theta$ from $\mathcal{N}(\bar{z}, 0.05^2)$, where $\bar{z}$ is itself drawn once per country. Private signals are $x_i = \theta + \varepsilon_i$, $\varepsilon_i \sim \mathcal{N}(0, \sigma^2)$ with $\sigma = 0.3$. The communication network is a Watts--Strogatz small-world graph with $k = 4$ neighbors and rewiring probability $p = 0.3$, regenerated each period. All random draws use NumPy's default \texttt{Generator} with no fixed seed; exact sequences are logged in per-run JSON files included in the replication archive.

\subsection{Code and Data Availability}

All code, prompts, cached LLM responses, and output data will be available at \texttt{[repository URL]} upon publication.

\end{document}
